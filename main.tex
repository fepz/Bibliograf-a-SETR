\documentclass{article}
\usepackage[utf8]{inputenc}
\usepackage[spanish]{babel}
\usepackage{geometry}
\usepackage{csquotes}
\usepackage[style=ieee]{biblatex}
\addbibresource{biblio.bib}

 \geometry{
    a4paper,
    total={170mm,257mm},
    left=20mm,
    top=20mm,
 }

\DeclareCiteCommand{\fullcite}
  {\usebibmacro{prenote}}
  {\clearfield{url}%
   \clearfield{pages}%
   \clearfield{pagetotal}%
   \clearfield{edition}%
   \clearfield{labelyear}%
   \usedriver
     {\DeclareNameAlias{sortname}{default}}
     {\thefield{entrytype}}}
  {\multicitedelim}
  {\usebibmacro{postnote}}

\title{Bibliografía Sistemas Embebidos y de Tiempo Real}
\author{Francisco Páez}

\begin{document}

\maketitle

\section{Introducción}
Este documento reúne los artículos, libros, reportes, etc., que forman parte de la bibliografía de la cátedra Sistemas Embebidos y de Tiempo Real. El objetivo es que sirva de guía para encontrar rápidamente el material de referencia para cada uno de los temas que se ven en la cursada.

\section{Principios de los Sistemas de Tiempo Real}
Estos trabajos presentan una introducción a los Sistemas de Tiempo Real:
\begin{itemize}
    \item \fullcite{Shin1994}
    \item \fullcite{Stankovic88}
    \item \fullcite{Urriza2008}
\end{itemize}

\section{Planificación de tareas periódicas}
\subsection*{Rate Monotonic}
\begin{itemize}
    \item \fullcite{LiuL73}
\end{itemize}

\subsection*{Deadline Monotonic}
\begin{itemize}
    \item \fullcite{LeungW82}
\end{itemize}

\subsection*{Earliest Deadline First}
\begin{itemize}
    \item \fullcite{LiuL73}
\end{itemize}

\subsection*{Prioridades limitadas}
\begin{itemize}
    \item \fullcite{CayssialsOSS99}
\end{itemize}

\section{Análisis de planificabilidad}
\subsection{Métodos por cotas}
\subsubsection*{Test de Liu \& Layland}
\begin{itemize}
    \item \fullcite{LiuL73}
\end{itemize}  
\subsubsection*{Test de Bini}
\begin{itemize}
    \item \fullcite{BiniBB01}
\end{itemize}    
    
\subsection{Métodos exactos}
Los siguientes trabajos describen métodos \emph{exactos}, esto es de \emph{condición necesaria y suficiente}, para la evaluación de planificabilidad bajo esquemas de asignación de prioridades fijas.
\subsubsection*{Test de Joseph \& Pandya}
\begin{itemize}
    \item \fullcite{JosephP86}
\end{itemize}
\subsubsection*{Mejora de Sjödin (RTA)}
\begin{itemize}
    \item \fullcite{SjodinH98}
\end{itemize}
\subsubsection*{Mejoras de Urriza (RTA2, RTA3, RTA4)}
\begin{itemize}
    \item \fullcite{Urriza2008b}
    \item \fullcite{Urriza2015}
\end{itemize}  

\section{Planificación de tareas aperiódicas}
\subsection*{Servidores}
\begin{itemize}
    \item \fullcite{StrosniderLS95}
    \item \fullcite{FohlerLB01}
    \item \fullcite{SpuriB94}
    \item \fullcite{SpruntLS88}
\end{itemize}
\subsection*{K-diagramabilidad}
\begin{itemize}
    \item \fullcite{Orozco2000}
\end{itemize}
\subsection*{Dual Priority}
\begin{itemize}
    \item \fullcite{DavisW95}
\end{itemize}
\subsection*{Slack Stealing}
\begin{itemize}
    \item \fullcite{Urriza2010}
    \item \fullcite{urriza2005fast}
\end{itemize}

\section{Recursos compartidos}
\subsection*{Herencia de prioridades y protocolo techo}
\begin{itemize}
    \item \fullcite{sha1990}
\end{itemize}

\subsection*{Stack Resource Policy}
\begin{itemize}
    \item \fullcite{Baker1990}
    \item \fullcite{Baker91}
\end{itemize}

\section{Ahorro de energía}
\begin{itemize}
    \item \fullcite{urriza2005b}
    \item \fullcite{urriza2009ahorro}
\end{itemize}

\section{Otros}
\begin{itemize}
    \item \fullcite{halang_contemporary_1993}
\end{itemize}

\end{document}
